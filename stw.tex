\documentclass[kpfonts]{patmorin}
\listfiles
\usepackage{pat}
\usepackage{paralist}
\usepackage{dsfont}  % for \mathds{A}
\usepackage[utf8x]{inputenc}
\usepackage{skull}

\usepackage{graphicx}
\usepackage[noend]{algorithmic}

\usepackage[normalem]{ulem}
\usepackage{cancel}
\usepackage{enumitem}

\usepackage[longnamesfirst,numbers,sort&compress]{natbib}

% Taken from
% https://tex.stackexchange.com/questions/42726/align-but-show-one-equation-number-at-the-end
\newcommand\numberthis{\addtocounter{equation}{1}\tag{\theequation}}


\setlength{\parskip}{1ex}


\DeclareMathOperator{\tw}{tw}
\DeclareMathOperator{\stw}{stw}
\DeclareMathOperator{\ltw}{ltw}
\DeclareMathOperator{\pw}{pw}
\DeclareMathOperator{\lpw}{lpw}
\DeclareMathOperator{\lhptw}{lhp-tw}
\DeclareMathOperator{\lhppw}{lhp-pw}


\title{\MakeUppercase{Some Notes on Simple Treewidth}}
\author{Vida Dujmović, Pat Morin, and David R. Wood}

\newcommand{\trn}{\chi_2}
\newcommand{\dtcn}{\bar{\chi}_2}
\newcommand{\htrn}{\hat{\chi}_2}
\newcommand{\scn}{\chi_{\star}}

\newtheorem{othertheorem}{Theorem}
\renewcommand*{\theothertheorem}{\Alph{othertheorem}}
\crefname{othertheorem}{Theorem}{Theorem}

\newtheoremstyle{named}{}{}{\itshape}{}{\bfseries}{.}{.5em}{#3}
\theoremstyle{named}
\newtheorem*{namedtheorem}{Unused}

\newcommand{\weirdref}[2]{\cref{#1}#2}
\newcommand{\weirdlabel}[2]{\label{#1-#1}}

\pagenumbering{roman}
\begin{document}
\begin{titlepage}
\maketitle

\begin{abstract}
  We present some basic results on simple treewidth.
\end{abstract}
\end{titlepage}

\tableofcontents

\newpage
\pagenumbering{arabic}

\section{Introduction}

A graph $H$ is a \emph{$t$-tree} if $H$ is a clique of size at most $t$ or if it contains a vertex $v$ such that $H[N_H(v)]$ is a $t$-clique and $H-\{t\}$ is a $t$-tree.  A \emph{$t$-forest} is a graph whose connected components are $t$-trees.

The recursive definition of $t$-trees implies that there is a permutation $\mathcal{R}:=(v_1,\ldots,v_n)$ of $V(H)$ such that, for each $i\in\{1,\ldots,n\}$, $H[N_H(v_i)\cap \{v_1,\ldots,v_{i-1}\}]$ is a clique of size $\min\{t,i-1\}$.  We call $\mathcal{R}$ a \emph{construction order} for $H$.  A construction order $\mathcal{R}$ induces a total order on $H$ that we denote by $<_{\mathcal{R}}$.

For two graphs $H$ and $X$, an \emph{$X$-decomposition} of $H$ is a sequence $\mathcal{X}:=(B_x:x\in V(X))$ of subsets of $V(H)$ called \emph{bags} indexed by the nodes of $X$ and such that
 \begin{inparaenum}[(i)]
     \item for each $v\in V(H)$, $X[\{x\in V(X):v\in B_x\}]$ is connected; and
     \item for each $vw\in E(H)$, there exists some $x\in V(X)$ such that $\{v,w\}\subseteq B_x$.
\end{inparaenum}
The \emph{width} of $\mathcal{X}$ is $\max\{|B_x|:x\in V(X)\}-1$.

In the special case where $X$ is a tree (or a forest), $\mathcal{X}$ is called a \emph{tree decomposition} of $H$.

It is not difficult to see that every $t$-tree $H$ has treewidth $t$ and a tree-decomposition of $H$ can be constructed incrementally from a construction order $v_1,\ldots,v_n$ of $H$: Start with a tree $T$ that consists of a single node $r$ with bag $B_r:=\{v_1,\ldots,v_{\min\{t+1,n\}}\}$ and, for $i\gets t+2,\ldots,n$, let $p_i$ be the minimum-depth node of $T$ such that $B_{p_i}$ contains the parent clique $N_H(v_i)\cap\{v_1,\ldots,v_{i-1}\}$ of $v_i$, and add a new leaf $x_i$ to $T$ adjacent to $p_i$ whose bag $B_{x_i}$ contains $v_i$ and its parent clique.  We call a tree-decomposition constructed this way a \emph{canonical} tree decomposition of $H$ generated by the construction order $v_1,\ldots,v_n$.

By construction, every bag of a canonical tree decomposition $\mathcal{T}$ of a $t$-tree $H$ has size exactly $\min\{t+1,|H|\}$, $T$ is rooted at $r$ and, for each node $x$ of $T$ with parent $p$, $|B_x\setminus B_p|=1$.
% Note that this implies that $|V(T)|=\max\{1,n-t\}$.
% , since $B_r$ contains $t+1$ vertices of $H$ and, for each $x\in V(T)\setminus\{r\}$, $B_x$ includes exactly one vertex of $H$ that is not in its parent $B_p$ where $p$ is the $T$-parent of $x$.
% For a rooted tree decomposition $\mathcal{T}:=(B_x:x\in V(T))$ of a graph $H$ and any $v\in V(H)$, we use the notation $x_T(v)$ to denote the minimum $T$-depth node $x\in V(T)$ such that $v\in B_x$.  This induces a partial order $\prec_{\mathcal{T}}$ on $V(H)$ in which $v\prec_{\mathcal{T}} w$ if and only if $x_T(v)\prec_T x_T(w)$.

% The following observation relates construction orders, orders induced by BFS layerings, and orders induced by canonical tree decompositions:
%
% \begin{obs}\label{order-relation}
%     Let $H$ be a $t$-tree with construction order $\mathcal{R}:=(v_1,\ldots,v_n)$ that generates a canonical tree decomposition $\mathcal{T}:=(B_x:x\in V(T))$ of $H$, and let $\mathcal{L}:=(L_0,\ldots,L_m)$ be a BFS layering of $H$ with $L_0:=\{v_1,\ldots,v_{\min\{n,t\}}\}$.  For any $v,w\in V(H)$,
%     if $v\prec_{\mathcal{T}} w$ or $v\prec_{\mathcal{L}} w$ then $v<_{\mathcal{R}} w$.
% \end{obs}
%
% \begin{obs}\label{dominant-parent}
%     Let $H$ be a $t$-tree with construction order $\mathcal{R}:=(v_1,\ldots,v_n)$ and let $\mathcal{L}:=(L_0,\ldots,L_m)$ be a BFS layering of $H$ with $L_0:=\{v_1,\ldots,v_{\min\{t,n\}}\}$.  Then, for any $i\in\{1,\ldots,m\}$ and any vertex $v\in L_{i}$ with dominant parent $p$, $p\in L_{i-1}$.
% \end{obs}
%
% \begin{proof}
%     A well-known property of BFS layerings is that, for $i>0$, any vertex $v\in L_{i}$ has some neighbour $w\in L_{i-1}$.  By definition, $p\le_\mathcal{R} w$ so, by \cref{order-relation} $p\le_\mathcal{L} w$. Since $w\in L_{i-1}$, $p\in L_0\cup\cdots\cup L_{i-1}$.  Another well known property of BFS layerings is that $v$ has no neighbour in $L_0,\ldots,L_{i-2}$.  Therefore $p\in L_{i-1}$.
% \end{proof}
%
% The following simple result is used several times in the (fairly involved) proof of \cref{simple-t-trees}.  We note that the vertices $p,q,v$ in the statement of the result are not necessarily distinct.
%
% \begin{obs}\label{up-clique}
%     Let $H$ be a $t$-tree with construction order $\mathcal{R}:=(v_1,\ldots,v_n)$, let $\mathcal{L}:=(L_0,\ldots,L_m)$ be a BFS layering of $H$ with $L_0:=\{v_1,\ldots,v_{\min\{t,n\}}\}$, let $i\in\{0,\ldots,m-1\}$, let $uvw$ be a path in $H$ with $u,w\in L_{i+1}$, $v\in L_{i}$, and let $p,q\in L_{i}$ be the dominant parents of $u$ and $w$ respectively.  Then $H[\{v,p,q\}]$ is a clique.
% \end{obs}
%
% % We remark that, in \cref{up-clique} $\{v,p,q\}$ may consist of $1$, $2$, or $3$ distinct elements.
%
% \begin{proof}
%     Refer to \cref{up-clique-figure}. Consider the canonical tree decomposition $\mathcal{T}:=(B_x:x\in V(T))$ of $H$ generated by $v_1,\ldots,v_n$.  By \cref{order-relation}, $v\prec_{\mathcal{T}} u,w$ and, by definition $p,q\preceq_{\mathcal{T}} v$.  Therefore $p,q\prec_{\mathcal{T}} v\prec_{\mathcal{T}} u,w$.  Since $pu\in E(H)$, this implies that $p\in B_{x_T(v)}$.  Similarly $q\in B_{x_T(v)}$.  By definition, $v\in B_{x_T(v)}$.  The vertices in $B_{x_T(v)}$ form a clique in $H$, so $\{v,p,q\}$ is a clique in $H$.
% \end{proof}
%
% \begin{figure}
%     \centering{
%         \includegraphics{figs/up-clique}
%     }
%     \caption{The proof of \cref{up-clique}}
%     \label{up-clique-figure}
% \end{figure}

\subsection{Simple Treewidth}


A tree decomposition $\mathcal{T}:=(B_x:x\in V(T))$ of a graph $H$ is \emph{$t$-simple} if it has width $t$ and, for every $t$-element subset $S\subseteq V(H)$, $|\{x\in V(T):S\subseteq B_x\}|\le 2$.  The simple treewidth $\stw(H)$ of a graph $H$ is the minimum integer $t$ such that $H$ has a $t$-simple tree decomposition \cite{knauer.ueckerdt:simple}.  A $t$-tree $H$ is a \emph{simple} $t$-tree if its canonical tree decompositions are $t$-simple.\footnote{Although a simple $t$-tree $H$ has different canonical tree decompositions depending on the choice of $v_1,\ldots,v_t$ in the construction order, each of these tree decompositions $(B_x:x\in V(T))$ generates the same set $\{B_x:x\in V(T)\}$ of bags that have a bijection with the $t+1$ cliques of $H$.  Therefore all of these tree decompositions are $t$-simple or none of them are.}  Though it appears implicitly, several times in the literature, \citet{knauer.ueckerdt:simple} define simple treewidth.  \citet{markenzon.justel.ea:subclasses} define simple $t$-trees (which they call Simple Clique (SC) $t$-trees).

Simple treewidth and treewidth are closely related:

\begin{lem}\label{simple-treewidth-vs-treewidth}\cite{knauer.ueckerdt:simple}
    For every graph $G$, $\tw(G)\le \stw(G)\le \tw(G)+1$.
\end{lem}

Despite \cref{simmple-treewidth-vs-treewidth}, simple treewidth and simple $t$-trees are worthy of separate study because they arise naturally:

\begin{lem}[\cite{knauer.ueckerdt:simple,markenzon.justel.ea:subclasses}]\label{simple-small-cases}
    A graph $H$ is
    \begin{compactenum}[(i)]
        \item a simple $1$-tree if and only if $H$ is a path.
        \item a simple $2$-tree if and only if $H$ is an edge-maximal outerplanar graph;
        \item a simple $3$-tree if and only if $H$ is a planar 3-tree.
    \end{compactenum}
\end{lem}

In addition to being a natural property of some graph classes, simple treewidth has recently proven to be useful for some applications \cite{X}. However, there are some properties of treewidth and $t$-trees that are easy to establish for which the corresponding property is not obvious for simple treewidth or simple $t$-trees.  One of these is that simple treewidth is a minor-closed:

\begin{lem}[\cite{knauer.ueckerdt:simple}]\label{simple-minor-closed}
    For every graph $G$ and every minor $M$ of $G$, $\stw(M)\le\stw(G)$.
\end{lem}

\cref{simple-minor-closed} is stated (without proof) by \citet{knauer.ueckerdt:simple}.

Another example of a property that is easy to prove for treewidth but more difficult for simple treewidth is the following:  Every graph of treewidth at most $t$ is the spanning subgraph of some $t$-tree.  For simple treewidth, the case $t=2$ is well known: Every outerplanar graph is a spanning subgraph of some edge-maximal outerplanar graph.  The case $t=3$, which states that every planar graph of treewidth at most 3 is the spanning subgraph of some $3$-tree, was proven by \citet{kratochvil.vaner:note}, who note that it is also implicit in earlier work of \citet{elmallah.colbourn:on}.  We are unaware of any proof for $t\ge 4$, so we give one here:

\begin{lem}\label{simple-subgraph}
    For any graph $G$ with $\stw(G)\le t$, there exists a simple $t$-tree $H$ with $V(G)= V(H)$ and $E(G)\subseteq E(H)$.
\end{lem}

\begin{proof}
    If $|G|\le t+1$, then the result is trivial, so suppose that $|G|>t+1$.
    Let $\mathcal{T}:=(B_x:x\in V(T))$ be a $t$-simple tree decomposition of $G$.  We will transform $\mathcal{T}$ into a canonical decomposition of an encompassing simple $t$-tree $H$.  Each step of this transformation maintains $t$-simplicity, i.e., after each step, $\mathcal{T}$ is a $t$-simple tree composition of some graph that contains $G$.

    We say that a node $x\in V(T)$ is \emph{redundant} if it has a neighbour $y$ such that $B_x\subseteq B_y$.  We say that $\mathcal{T}$ is \emph{concise} if $T$ has no redundant nodes.  We may assume that $\mathcal{T}$ is concise since,  if $B_x\subseteq B_y$ for some $xy\in E(T)$ then we can contract $x$ into $y$ without affecting $t$-simplicity.  Repeating this exhaustively ensures that $\mathcal{T}$ is concise.

    Next, we do transformations on $T$ (if necessary) so that $|B_r|=t+1$ for at least one node $r\in V(T)$. Let $r$ be a node in $T$ that maximizes $|B_r|$. As long as $|B_r|<t+1$ we find a neighbour $x$ of $r$ and add an element of $B_x\setminus B_r$ to $B_r$.  If this causes $B_x$ to become redundant (because $B_x\subset B_r$) the we contract $x$ into $r$.  To see that this operation maintains $t$-simplicity, suppose that, after the operation some $t$-element subset of $V(G)$ appears in $B_r$, $B_x$, and $B_y$ for some node $y\neq r$ adjacent to $x$.  Since $B_y$ and $B_x$ are not changed by the operation, $C\subseteq B_x\cap B_y$.  But $|B_y|\le t$ and $\mathcal{T}$ is concise, so $|C|\le|B_x\cap B_y|\le|B_y|-1= t-1$, a contradiction. Thus we may assume that $T$ is rooted at a node $r$ with $|B_r|=t+1$.

    Now, we say that $\mathcal{T}$ is \emph{maximally deep} if no node $x$ has two children $y$ and $z$ with $B_y\cap B_x\subseteq B_z\cap B_x$.  If such a configuration of nodes exists, then we can remove the edge $xy$ and replace it with the edge $zy$, to make $y$ a child of $z$.  This operation does not affect $t$-simplicity and increases the \emph{total path length} $\sum_{x\in V(T)} d_T(x)$, which is upper bounded by $\binom{|T|}{2}$ and, since $T$ is concise and has at least one node $r$ with $|B_r|=t+1$, $|T|\le n-t$.  Therefore after $O(|G|^2)$ such operations, $\mathcal{T}$ is maximally deep.

    We say that a node $x\in V(T)$ with parent $p$ is \emph{incremental} if $|B_x\setminus B_y|=1$ and use the convention that the root $r$ is always incremental.  We say that $x$ is \emph{maximal} if $|B_x|=t+1$.  We say that $\mathcal{T}$ is \emph{uniform} if, for each $x\in V(T)\setminus\{r\}$, $x$ is incremental and maximal.  Observe that, if $\mathcal{T}$ is uniform then $\mathcal{T}$ is a canonical tree decomposition of a simple $t$-tree $H$, and we are done.  For the remainder of this proof, every operation we perform will increase the \emph{total bag size} of $\mathcal{T}$, defined as $\bigcup_{x\in V(T)} |B_x|$.  Note that, for any concise tree decomposition of $\mathcal{G}$, the total bag size is at most $(t+1)|G|$ since, for each non-root node $x$ with parent $p$, $B_x$ contains a vertex $v$ not contained in $B_p$.

    If $\mathcal{T}$ is not uniform, then there exists $x\in V(T)$ (possibly $x=r$) such that every $T$-ancestor of $x$, including $x$, is incremental and maximal, but some child $y$ of $x$ is not incremental or not maximal.  Observe that $|B_x\cap B_y| < t$, otherwise $y$ is redundant (because $|B_y|\le |B_x\cap B_y|$ so $B_y=B_x\cap B_y$) or $y$ is maximal and incremental (because $|B_y|=t+1$ so $|B_y\setminus B_x|=|B_y|-|B_x\cap B_y|=1$).

    % Consider the subtree $T_y$ of $T$ containing all $T$-descendants of $y$, including $y$.  $T_y$ has a subtree $T_y'$ containing all nodes $u$ such that $B_u\supseteq B_x\cap B_y$.  The tree $T_y'$ is non-empty and therefore contains some node $u$ of degree at most one.  We replace the edge $xy$ in $T$ with $xu$.  This does not change the fact that $\mathcal{T}$ is a $t$-simple tree decomposition of $G$.  Setting $y:=u$ allows us to continue, using the property that $y$ has at most one child $z$ with $B_x\cap B_y\subseteq B_z$.

    Since $|B_x\cap B_y|< t$ and $|B_x|=t+1$, $B_x\setminus B_y$ contains at least two elements $v$ and $w$.  At least one of the following is true:
    \begin{inparaenum}[(i)]
        \item $x=r$ is the root of $T$; or
        \item at least one of $v$ or $w$, say $v$, is not contained in $B_p$ where $p$ is the parent of $x$.
    \end{inparaenum}
    % If $|B_y|\le t$, then we add $v$ to $B_y$.
    % We claim that this preserves the $t$-simplicity of $\mathcal{T}$.
    We claim that there is no path $xyz$ in $T$ such that $B_z\supseteq (B_x\cap B_y)\cup\{v\}$. To prove this, there are three cases to consider:
    \begin{compactenum}
        \item $z$ is a child of $x$:  This is not possible since $\mathcal{T}$ is maximally deep, so $B_x\cap B_y\not\subseteq B_z$.
        \item $z$ is a child of $y$:  This is not possible since, before the operation, $v\not\in B_y$, and $v\in B_x$, so $v\not\in B_z$.
        \item $z$ is the parent of $x$:  This is not possible since $v$ was deliberately chosen to avoid this.
    \end{compactenum}
    % This operation increases the total size of all bags $\mathcal{T}$ so we can continue.

    If $|B_y|\le t$ then we add $v$ to $B_y$.  This operation increases the total bag size of $\mathcal{T}$, and this step is complete.  That this operation preserves $t$-simplicity follows from the non-existence of the path $xyz$ in the previous paragraph.

    If $|B_y|=t+1$ and $|B_x\cap B_y|\le t-2$ then we introduce a new node $x'$ with $B_{x'}=B_y\cup\{v\}$ and replace the edge $xy$ with the path $xx'y$.  This preserves $t$-simplicity since $|B_{x'}|\le t-1<t$, so $B_{x'}$ contains no $t$-element subsets of $V(G)$.

    We are finally left with the case where $|B_y|=t+1$ and $|B_x\cap B_y|= t-1$.  Consider the subtree $T_y$ of $T$ containing all $T$-descendants of $y$, including $y$.  $T_y$ has a subtree $T_y'$ containing all nodes $u$ such that $B_u\supseteq B_x\cap B_y$.  The tree $T_y'$ is non-empty and therefore contains some node $z$ of degree at most one.  We replace the edge $xy$ in $T$ with $xz$.  This does not change the fact that $\mathcal{T}$ is a $t$-simple tree decomposition of $G$.  Setting $y:=z$ allows us to continue, now assuming that $y$ has at most one child $z$ with $B_z\supseteq B_x\cap B_y$.

    Since $|B_x\cap B_y|<t$ and $|B_y|=t+1$, $B_y\setminus B_x$ contains at least two elements $w$ and $\xi$.  At least one of the following is true:
    \begin{inparaenum}[(i)]
        \item $y$ has no child $z$ such that $B_z\supseteq B_x\cap B_y$;
        \item $y$ has a child $z$ such that $B_z\supseteq B_x\cap B_y$ and at least one of $w$ or $\xi$, say $w$, is not contained in $B_z$.
    \end{inparaenum}
    In this case, we introduce a new node $x'$ with $B_{x'}:=(B_x\cap B_y)\cup\{v,w\}$ and replace the edge $xy$ with the path $xx'y$.  We claim that this operation preserves $t$-simplicity.  To see this, we need to check three conditions:
    \begin{compactenum}
        \item The $t$-element set $(B_x\cap B_y)\cup\{v\}$ is contained in $B_x$ and $B_{x'}$ only.
        \item The $t$-element set $(B_x\cap B_y)\cup\{w\}$ is contained in $B_{x'}$ and $B_y$ only.
        \item Any $t$-element subset of $B_{x'}$ that includes both $v$ and $w$ appears only in $B_{x'}$.
    \end{compactenum}
    Again, this operation increases the total bag size of $\mathcal{T}$, so this step is complete.
    % This operation increases the total size of all bags in $\mathcal{T}$ and therefore represents progress.
    %
    %
    %
    %     $\{v,w\}$
    %
    % % If $|
    %
    %
    %
    %
    %
    % If $T_y'$ contains a node $z$ with $|B_z|<t+1$, then we do this and the algorithm continues.  Otherwise, $T_y'$
    % %
    %
    % (Assume that $|B_x\cap B_y|=t-1$, otherwise things are easy.)
    %
    % If $|B_y|=t+1$ then, let $C:=(B_x\cap B_y)\cup \{v\}$ where $v$ is defined as above (so $v$ does not appear in the bag of the parent of $x$).  Let $\{v_1,\ldots,v_d\}:=B_y\setminus B_x$.  Since $|B_x\cap B_y|<t$, $d\ge 2$.
    %
    %
    %  There are some cases to consider:
    %
    % \begin{compactenum}
    %
    %     \item If $y$ has a child $z$ such that $v_1\cup (B_x\cap B_y)\not\subseteq B_z$, then we replace $xy$ with a path $xx'y$ where $B_{x'}:C\cup\{v_1\}$.  This preserves $t$-simplicity since the only new $t$-clique to worry about is $v_1\cup (B_x\cap B_y)$ but this only appears in $B_y$ (and now in $B_{x'}$).
    %
    %     \item Otherwise, $y$ has exactly one child $z$ such that $v_1\in B_z$.  Now, observe that $C':=(B_x\cap B_y)\cup\{v_1\}$ is contained in $B_y$ and $B_z$.  Therefore,
    %
    %
    % arguing as above, there is a vertex $v$ that appears only in $B_x$.
\end{proof}

\begin{lem}\label{simple-bfs-layers}
    Let $H$ be a simple $t$-tree, let $n:=|H|$, let $v_1,\ldots,v_n$ be a construction order for $H$, and let $L_0,\ldots,L_m$ be the BFS ordering of $H$ with $L_0:=v_1,\ldots,v_{\min\{t,n\}}$.   Then, for each $i\in\{1,\ldots,m\}$, $\stw(H[L_i])\le t-1$.
\end{lem}

\begin{proof}
    Delete BFS layers $L_{i+1},\ldots,L_m$ and contract BFS layers $L_0,\ldots,L_{i-1}$ into a single vertex $v_0$ to obtain a minor $H'$ of $H$ in which $v_0$ is a dominant vertex.  By \cref{simple-minor-closed}, $\stw(H')\le t$, so $H'$ has a $t$-simple tree decomposition $\mathcal{T}':=(B_x:x\in V(T))$.  Since $v_0$ is a dominant vertex, $v_0$ appears in every bag of $\mathcal{T}'$.  Therefore $\mathcal{T}:=(B_x\setminus\{v_0\}:x\in V(T))$ is a $(t-1)$-simple tree decomposition of $H'-\{v_0\}=H[L_i]$.
\end{proof}


\bibliographystyle{plainnat}
\bibliography{us}

\end{document}
