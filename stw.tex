\documentclass[kpfonts]{patmorin}
\listfiles
\usepackage{pat}
\usepackage{paralist}
\usepackage{dsfont}  % for \mathds{A}
\usepackage[utf8x]{inputenc}
\usepackage{skull}

\usepackage{graphicx}
\usepackage[noend]{algorithmic}

\usepackage[normalem]{ulem}
\usepackage{cancel}
\usepackage{enumitem}

\usepackage[longnamesfirst,numbers,sort&compress]{natbib}

% Taken from
% https://tex.stackexchange.com/questions/42726/align-but-show-one-equation-number-at-the-end
\newcommand\numberthis{\addtocounter{equation}{1}\tag{\theequation}}


\setlength{\parskip}{1ex}


\DeclareMathOperator{\tw}{tw}
\DeclareMathOperator{\stw}{stw}
\DeclareMathOperator{\ltw}{ltw}
\DeclareMathOperator{\pw}{pw}
\DeclareMathOperator{\lpw}{lpw}
\DeclareMathOperator{\lhptw}{lhp-tw}
\DeclareMathOperator{\lhppw}{lhp-pw}


\title{\MakeUppercase{Notes on Simple Treewidth}}
\author{Vida Dujmović, Pat Morin, and David R. Wood}

\newcommand{\trn}{\chi_2}
\newcommand{\dtcn}{\bar{\chi}_2}
\newcommand{\htrn}{\hat{\chi}_2}
\newcommand{\scn}{\chi_{\star}}

\newtheorem{othertheorem}{Theorem}
\renewcommand*{\theothertheorem}{\Alph{othertheorem}}
\crefname{othertheorem}{Theorem}{Theorem}

\newtheoremstyle{named}{}{}{\itshape}{}{\bfseries}{.}{.5em}{#3}
\theoremstyle{named}
\newtheorem*{namedtheorem}{Unused}

\newcommand{\weirdref}[2]{\cref{#1}#2}
\newcommand{\weirdlabel}[2]{\label{#1-#1}}

\begin{document}
\maketitle

\begin{abstract}
  We present some basic results on simple treewidth.
\end{abstract}

\section{Introduction}

Formalized by \citet{knauer.ueckerdt:simple} and studied thoroughly in the thesis of \citet{wulf:stacked} (under a different name), \emph{simple treewidth} is a more restrictive and less well-known variant of treewidth.\footnote{Though \citet{knauer.ueckerdt:simple} were the first to formalize a general definition, simple treewidth was used implicitly several times prior to this.}  Just as treewidth is related to $k$-trees, simple treewidth is related to \emph{simple $k$-trees}, which were defined and studied by \citet{markenzon.justel.ea:subclasses} (who call them Simple Clique (SC) $k$-trees).

The purpose of the present note is to collect a few useful results about simple treewidth.  This is motivated by two instances \cite{X,Y} where simple treewidth served a helpful role, but basic properties of simple treewidth needed to be rediscovered (or at least verified) because these properties have not been documented (or proven) anywhere.  A glaring example of this is the fact that every graph of simple treewidth $k$ is a spanning subgraph of a simple $k$-tree.  The equivalent statement for treewidth and $k$-trees is very well-known and easy to prove. For simple treewidth, even the case $k=3$ is non-trivial \cite{kratochvil.vaner:note,elmallah.colbourn:on}.  For $k\ge 4$, the result can be deduced from a result (Theorem 3.27) in the thesis of \citet{wulf:stacked}, but we are unaware of any complete proof of (or explicit statement of) this fact for $k\ge 4$.  [TODO: Look more carefully at and around Theorem~3.27 in Wulf's thesis, which shows that every graph of simple treewidth $k$ is contained in some simple $k$-tree.]

Indeed, some properties of treewidth and $k$-trees that are easy to establish are considerably more difficult to establish for simple treewidth or simple $k$-trees, and this discourages the use of simple treewidth, even when it is the appropriate parameter.  The authors hope that this note will help correct this by encouraging the use of simple treewidth in applications where it is appropriate, as doing so can lead to results that are more general.  For example, there are results that are true for graphs of treewidth 1 (trees) that are just as easy to prove for graphs of simple treewidth 2 (outerplanar graphs).  Proving them this way immediately gives a more general result.

Finally, we emphasize that many of the results presented here are known. In cases where a proof of the result is available, we simply cite it.  For results stated without proof, we provide a proof, but reference the original statement of the result.  The remainder of the paper has two sections: \cref{treewidth} reviews definitions of treewidth and $k$-trees. \cref{simple-treewidth} defines simple treewidth, simple $k$-trees, and proves basic results about both these parameters.


\section{Treewidth}
\label{treewidth}

A graph $H$ is a \emph{$t$-tree} if $H$ is a clique of size at most $t$ or if it contains a vertex $v$ such that $H[N_H(v)]$ is a $t$-clique and $H-\{t\}$ is a $t$-tree.  A \emph{$t$-forest} is a graph whose connected components are $t$-trees.

The recursive definition of $t$-trees implies that there is a permutation $\mathcal{R}:=(v_1,\ldots,v_n)$ of $V(H)$ such that, for each $i\in\{1,\ldots,n\}$, $H[N_H(v_i)\cap \{v_1,\ldots,v_{i-1}\}]$ is a clique of size $\min\{t,i-1\}$.  We call $\mathcal{R}$ a \emph{construction order} for $H$.  A construction order $\mathcal{R}$ induces a total order on $H$ that we denote by $<_{\mathcal{R}}$.

A \emph{tree decomposition} of a graph $G$ is a sequence $\mathcal{T}:=(B_x:x\in V(T))$ of subsets of $V(G)$ called \emph{bags} indexed by the nodes of a tree $T$ such that
 \begin{inparaenum}[(i)]
     \item for each $v\in V(G)$, $T[\{x\in V(T):v\in B_x\}]$ is connected; and
     \item for each $vw\in E(G)$, there exists some $x\in V(T)$ such that $\{v,w\}\subseteq B_x$.
\end{inparaenum}
The \emph{width} of $\mathcal{T}$ is $\max\{|B_x|:x\in V(T)\}-1$.  The \emph{treewidth}, $\tw(G)$ is the minimum integer $k$ such that $G$ has a tree decomposition of width $k$.

Property~(i) of tree decompositions says that each vertex of $G$ is associated with a (connected) subtree of $T$.  The following well-known lemma follows from the $1$-Helly property of subtrees:

\begin{lem}
\label{TreeDecompositionClique}
    For each clique $C$ of a graph $G$, each tree decomposition $\mathcal{T}:=(B_x:x\in V(T))$ of $G$ has a bag $B_x\supseteq C$ containing $C$.
\end{lem}


[TODO: $H\Rightarrow G$ throughout and $t\Rightarrow k$ throughout.]

It is not difficult to see that every $t$-tree $H$ has treewidth $t$ and a tree-decomposition of $H$ can be constructed incrementally from a construction order $v_1,\ldots,v_n$ of $H$: Start with a tree $T$ that consists of a single node $r$ with bag $B_r:=\{v_1,\ldots,v_{\min\{t+1,n\}}\}$ and, for $i\gets t+2,\ldots,n$, let $p_i$ be the minimum-depth node of $T$ such that $B_{p_i}$ contains the parent clique $N_H(v_i)\cap\{v_1,\ldots,v_{i-1}\}$ of $v_i$, and add a new leaf $x_i$ to $T$ adjacent to $p_i$ whose bag $B_{x_i}$ contains $v_i$ and its parent clique.  We call a tree-decomposition constructed this way a \emph{canonical} tree decomposition of $H$ generated by the construction order $v_1,\ldots,v_n$.

By construction, every bag of a canonical tree decomposition $\mathcal{T}$ of a $t$-tree $H$ has size exactly $\min\{t+1,|H|\}$, $T$ is rooted at $r$ and, for each edge $xy\in E(T)$, $|B_x\setminus B_p|=1$.  Indeed, for any canonical tree decomposition of $H$, there is a bijection between the $(t+1)$-cliques of $H$ and the bags of the decomposition.

A graph $G$ is \emph{chordal} if it has no induced cycle of length greater than 3.  It is well known that $G$ is chordal if and only if $V(G)$ has an ordering $v_1,\ldots,v_n$ such that, for each $i\in\{2,\ldots,n\}$, $G[N_G(v_i)\cap\{v_1,\ldots,v_{i-1}]$ is a clique.  Clearly any $t$-tree is chordal.  Building a canonical tree decomposition of $G$ using the ordering $v_1,\ldots,v_n$ proves one direction of the following well-known result (the other direction follows from a preorder traversal of $T$):

% Note that this implies that $|V(T)|=\max\{1,n-t\}$.
% , since $B_r$ contains $t+1$ vertices of $H$ and, for each $x\in V(T)\setminus\{r\}$, $B_x$ includes exactly one vertex of $H$ that is not in its parent $B_p$ where $p$ is the $T$-parent of $x$.
% For a rooted tree decomposition $\mathcal{T}:=(B_x:x\in V(T))$ of a graph $H$ and any $v\in V(H)$, we use the notation $x_T(v)$ to denote the minimum $T$-depth node $x\in V(T)$ such that $v\in B_x$.  This induces a partial order $\prec_{\mathcal{T}}$ on $V(H)$ in which $v\prec_{\mathcal{T}} w$ if and only if $x_T(v)\prec_T x_T(w)$.

% The following observation relates construction orders, orders induced by BFS layerings, and orders induced by canonical tree decompositions:
%
% \begin{obs}\label{order-relation}
%     Let $H$ be a $t$-tree with construction order $\mathcal{R}:=(v_1,\ldots,v_n)$ that generates a canonical tree decomposition $\mathcal{T}:=(B_x:x\in V(T))$ of $H$, and let $\mathcal{L}:=(L_0,\ldots,L_m)$ be a BFS layering of $H$ with $L_0:=\{v_1,\ldots,v_{\min\{n,t\}}\}$.  For any $v,w\in V(H)$,
%     if $v\prec_{\mathcal{T}} w$ or $v\prec_{\mathcal{L}} w$ then $v<_{\mathcal{R}} w$.
% \end{obs}
%
% \begin{obs}\label{dominant-parent}
%     Let $H$ be a $t$-tree with construction order $\mathcal{R}:=(v_1,\ldots,v_n)$ and let $\mathcal{L}:=(L_0,\ldots,L_m)$ be a BFS layering of $H$ with $L_0:=\{v_1,\ldots,v_{\min\{t,n\}}\}$.  Then, for any $i\in\{1,\ldots,m\}$ and any vertex $v\in L_{i}$ with dominant parent $p$, $p\in L_{i-1}$.
% \end{obs}
%
% \begin{proof}
%     A well-known property of BFS layerings is that, for $i>0$, any vertex $v\in L_{i}$ has some neighbour $w\in L_{i-1}$.  By definition, $p\le_\mathcal{R} w$ so, by \cref{order-relation} $p\le_\mathcal{L} w$. Since $w\in L_{i-1}$, $p\in L_0\cup\cdots\cup L_{i-1}$.  Another well known property of BFS layerings is that $v$ has no neighbour in $L_0,\ldots,L_{i-2}$.  Therefore $p\in L_{i-1}$.
% \end{proof}
%
% The following simple result is used several times in the (fairly involved) proof of \cref{simple-t-trees}.  We note that the vertices $p,q,v$ in the statement of the result are not necessarily distinct.
%
% \begin{obs}\label{up-clique}
%     Let $H$ be a $t$-tree with construction order $\mathcal{R}:=(v_1,\ldots,v_n)$, let $\mathcal{L}:=(L_0,\ldots,L_m)$ be a BFS layering of $H$ with $L_0:=\{v_1,\ldots,v_{\min\{t,n\}}\}$, let $i\in\{0,\ldots,m-1\}$, let $uvw$ be a path in $H$ with $u,w\in L_{i+1}$, $v\in L_{i}$, and let $p,q\in L_{i}$ be the dominant parents of $u$ and $w$ respectively.  Then $H[\{v,p,q\}]$ is a clique.
% \end{obs}
%
% % We remark that, in \cref{up-clique} $\{v,p,q\}$ may consist of $1$, $2$, or $3$ distinct elements.
%
% \begin{proof}
%     Refer to \cref{up-clique-figure}. Consider the canonical tree decomposition $\mathcal{T}:=(B_x:x\in V(T))$ of $H$ generated by $v_1,\ldots,v_n$.  By \cref{order-relation}, $v\prec_{\mathcal{T}} u,w$ and, by definition $p,q\preceq_{\mathcal{T}} v$.  Therefore $p,q\prec_{\mathcal{T}} v\prec_{\mathcal{T}} u,w$.  Since $pu\in E(H)$, this implies that $p\in B_{x_T(v)}$.  Similarly $q\in B_{x_T(v)}$.  By definition, $v\in B_{x_T(v)}$.  The vertices in $B_{x_T(v)}$ form a clique in $H$, so $\{v,p,q\}$ is a clique in $H$.
% \end{proof}
%
% \begin{figure}
%     \centering{
%         \includegraphics{figs/up-clique}
%     }
%     \caption{The proof of \cref{up-clique}}
%     \label{up-clique-figure}
% \end{figure}


\begin{lem}
\label{TreeDecompositionChordal}
    A graph $G$ is chordal if and only if it has a tree-decomposition $\mathcal{T}:=(B_x:x\in V(T))$ in which $B_x$ is a clique in $G$ for each $x\in V(T))$.
\end{lem}


\subsection{Simple Treewidth}
\label{simple-treewidth}

A tree decomposition $\mathcal{T}:=(B_x:x\in V(T))$ of a graph $H$ is \emph{$t$-simple} if it has width at most $t$ and, for every $t$-element subset $S\subseteq V(H)$, $|\{x\in V(T):S\subseteq B_x\}|\le 2$.  The simple treewidth $\stw(H)$ of a graph $H$ is the minimum integer $t$ such that $H$ has a $t$-simple tree decomposition \cite{knauer.ueckerdt:simple}.  A $t$-tree $H$ is a \emph{simple} $t$-tree if its canonical tree decompositions are $t$-simple.\footnote{Although a simple $t$-tree $H$ has different canonical tree decompositions depending on the choice of $v_1,\ldots,v_t$ in the construction order, each of these tree decompositions $(B_x:x\in V(T))$ generates the same set $\{B_x:x\in V(T)\}$ of bags that have a bijection with the $t+1$ cliques of $H$.  Therefore all of these tree decompositions are $t$-simple or none of them are.}


\begin{lem}[\cite{knauer.ueckerdt:simple}]\label{simple-treewidth-vs-treewidth}
    For every graph $G$, $\tw(G)\le \stw(G)\le \tw(G)+1$.
\end{lem}

Although \cref{simple-treewidth-vs-treewidth} suggests there is little difference between treewidth and simple treewidth, simple $t$-trees arise naturally in a number of settings and are related to the Colin de Verdière number $\mu(H)$ of a graph $H$:

\begin{lem}[\cite{knauer.ueckerdt:simple,markenzon.justel.ea:subclasses,fallat.mitchell:colin}]\label{simple-small-cases}
    A graph $H$ is
    \begin{compactenum}[(i)]
        \item a simple $1$-tree if and only if $H$ is a path.
        \item a simple $2$-tree if and only if $H$ is an edge-maximal outerplanar graph;
        \item a simple $3$-tree if and only if $H$ is a planar 3-tree.
        \item a simple $4$-tree if and only if $H$ is a linkelessly embeddable 4-tree.
        \item a simple $k$-tree if and only if $H$ is a $k$-tree and $\mu(H)=k$.
    \end{compactenum}
\end{lem}

An extremely useful property of simple treewidth is that it is minor-monotone, a fact that is stated without proof by \citet{knauer.ueckerdt:simple}.  A complete proof is given by \citet[Theorem~5.2]{wulf:stacked}.

 % and whose proof requires some case analysis.

\begin{lem}[\cite{knauer.ueckerdt:simple,wulf:stacked}]\label{simple-minor-closed}
    For every graph $G$ and every minor $M$ of $G$, $\stw(M)\le\stw(G)$.
\end{lem}

The following is another example of a property that is easy to prove for treewidth but more difficult to establish for simple treewidth:

 % is the following:  Every graph of treewidth at most $t$ is the spanning subgraph of some $t$-tree.  For simple treewidth, the case $t=2$ is well known: Every outerplanar graph is a spanning subgraph of some edge-maximal outerplanar graph.  The case $t=3$, which states that every planar graph of treewidth at most 3 is the spanning subgraph of some $3$-tree, was proven by \citet{kratochvil.vaner:note}, who note that it is also implicit in earlier work of \citet{elmallah.colbourn:on}.  We are unaware of any proof, or statement, of this for $t\ge 4$, so we give one here:

\begin{lem}\label{simple-subgraph}
    For any graph $G$ with $\stw(G)\le t$, there exists a simple $t$-tree $H$ with $V(G)= V(H)$ and $E(G)\subseteq E(H)$.
\end{lem}

\begin{proof}
    If $|G|\le t+1$, then the result is trivial, so suppose that $|G|>t+1$. Let $\mathcal{T}:=(B_x:x\in V(T))$ be a $t$-simple tree decomposition of $G$. We will transform $\mathcal{T}$ into a canonical decomposition of an encompassing simple $t$-tree $H$. Each step of this transformation maintains $t$-simplicity, i.e., after each step, $\mathcal{T}$ is a $t$-simple tree decomposition of some graph that contains $G$.

    We say that a node $x\in V(T)$ is \emph{redundant} if it has a neighbour $y$ such that $B_x\subseteq B_y$.  We say that $\mathcal{T}$ is \emph{concise} if $T$ has no redundant nodes.  We may assume that $\mathcal{T}$ is concise since,  if $B_x\subseteq B_y$ for some $xy\in E(T)$ then we can contract $x$ into $y$ without affecting $t$-simplicity.  Repeating this exhaustively ensures that $\mathcal{T}$ is concise.

    Next, we do transformations on $T$ (if necessary) so that $|B_r|=t+1$ for at least one node $r\in V(T)$. Let $r$ be a node in $T$ that maximizes $|B_r|$. As long as $|B_r|<t+1$ we find a neighbour $x$ of $r$ and add an element of $B_x\setminus B_r$ to $B_r$.  If this causes $B_x$ to become redundant (because $B_x\subset B_r$) the we contract $x$ into $r$.  To see that this operation maintains $t$-simplicity, suppose that, after the operation some $t$-element subset of $V(G)$ appears in $B_r$, $B_x$, and $B_y$ for some node $y\neq r$ adjacent to $x$.  Since $B_y$ and $B_x$ are not changed by the operation, $C\subseteq B_x\cap B_y$.  But $|B_y|\le t$ and $\mathcal{T}$ is concise, so $|C|\le|B_x\cap B_y|\le|B_y|-1= t-1$, a contradiction. Thus we may assume that $T$ is rooted at a node $r$ with $|B_r|=t+1$.

    Now, we say that $\mathcal{T}$ is \emph{maximally deep} if no node $x$ has two children $y$ and $z$ with $B_y\cap B_x\subseteq B_z\cap B_x$.  Refer to \cref{maximal}.  If such a configuration of nodes exists, then we can remove the edge $xy$ and replace it with the edge $zy$, to make $y$ a child of $z$.  This operation does not affect $t$-simplicity and increases the \emph{total path length} $\sum_{x\in V(T)} d_T(x)$, which is upper bounded by $\binom{|T|}{2}$ and, since $T$ is concise and has at least one node $r$ with $|B_r|=t+1$, $|T|\le n-t$.  Therefore after $O(|G|^2)$ such operations, $\mathcal{T}$ is maximally deep.

    \begin{figure}
        \centering{
            \begin{tabular}{c@{\hspace{1cm}}c}
                \includegraphics{figs/maximal-1} &
                % \vspace{-2cm} &
                \includegraphics{figs/maximal-2}
            \end{tabular}
        }
        \caption{Making $\mathcal{T}$ maximally deep.}
        \label{maximal}
    \end{figure}

    We say that a node $x\in V(T)$ with parent $p$ is \emph{incremental} if $|B_x\setminus B_y|=1$ and use the convention that the root $r$ is always incremental.  We say that $x$ is \emph{maximal} if $|B_x|=t+1$.  We say that $\mathcal{T}$ is \emph{uniform} if, for each $x\in V(T)\setminus\{r\}$, $x$ is incremental and maximal.  Observe that, if $\mathcal{T}$ is uniform then $\mathcal{T}$ is a canonical tree decomposition of a simple $t$-tree $H$, and we are done.  For the remainder of this proof, every operation we perform will increase the \emph{total bag size} of $\mathcal{T}$, defined as $\bigcup_{x\in V(T)} |B_x|$.  Note that, for any concise tree decomposition of $\mathcal{G}$, the total bag size is at most $(t+1)|G|$ since, for each non-root node $x$ with parent $p$, $B_x$ contains a vertex $v$ not contained in $B_p$.

    If $\mathcal{T}$ is not uniform, then there exists $x\in V(T)$ (possibly $x=r$) such that every $T$-ancestor of $x$, including $x$, is incremental and maximal, but some child $y$ of $x$ is not incremental or not maximal.  Observe that $|B_x\cap B_y| < t$, otherwise $y$ is redundant (because $|B_y|\le |B_x\cap B_y|$ so $B_y=B_x\cap B_y$) or $y$ is maximal and incremental (because $|B_y|=t+1$ so $|B_y\setminus B_x|=|B_y|-|B_x\cap B_y|=1$).

    % Consider the subtree $T_y$ of $T$ containing all $T$-descendants of $y$, including $y$.  $T_y$ has a subtree $T_y'$ containing all nodes $u$ such that $B_u\supseteq B_x\cap B_y$.  The tree $T_y'$ is non-empty and therefore contains some node $u$ of degree at most one.  We replace the edge $xy$ in $T$ with $xu$.  This does not change the fact that $\mathcal{T}$ is a $t$-simple tree decomposition of $G$.  Setting $y:=u$ allows us to continue, using the property that $y$ has at most one child $z$ with $B_x\cap B_y\subseteq B_z$.

    Refer to \cref{tough-one}.
    Since $|B_x\cap B_y|< t$ and $|B_x|=t+1$, $B_x\setminus B_y$ contains at least two elements $v$ and $\eta$.  Since $x$ is incremental and maximal, at least one of the following is true:
    \begin{inparaenum}[(i)]
        \item $x=r$ is the root of $T$; or
        \item $x$ has a parent $p$ and at least one of $v$ or $\eta$, say $v$, is not contained in $B_p$.
    \end{inparaenum}
    % If $|B_y|\le t$, then we add $v$ to $B_y$.
    % We claim that this preserves the $t$-simplicity of $\mathcal{T}$.
    We claim that there is no path $xyz$ in $T$ such that $B_z\supseteq (B_x\cap B_y)\cup\{v\}$. To prove this, there are three cases to consider:
    \begin{compactenum}
        \item $z$ is a child of $x$:  This is not possible since $\mathcal{T}$ is maximally deep, so $B_x\cap B_y\not\subseteq B_z$.
        \item $z$ is a child of $y$:  This is not possible since $v\in B_x$ and $v\not\in B_y$, so $v\not\in B_z$.
        \item $z$ is the parent of $x$:  This is not possible since $v$ was deliberately chosen to avoid this.
    \end{compactenum}
    % This operation increases the total size of all bags $\mathcal{T}$ so we can continue.
    \begin{figure}
        \centering{
            \begin{tabular}{c@{\hspace{2cm}}c}
                \includegraphics{figs/transform-1} &
                % \vspace{-2cm} &
                \includegraphics{figs/transform-2}
            \end{tabular}
        }
        \caption{Making $\mathcal{T}$ maximally deep.}
        \label{tough-one}
    \end{figure}

    If $|B_y|\le t$ then we add $v$ to $B_y$.  This operation increases the total bag size of $\mathcal{T}$, and this step is complete.  That this operation preserves $t$-simplicity follows from the non-existence of the path $xyz$ in the previous paragraph.  In particular, before this operation, the $t$-element subset $(B_x\cap B_y)\cup\{v\}$ appears only in $B_x$. After this operation, it appears only in $B_x$ and $B_y$.

    % If $|B_y|=t+1$ and $|B_x\cap B_y|\le t-2$ then we introduce a new node $x'$ with $B_{x'}=(B_x\cap B_y)\cup\{v,w\}$ where $w$ is an arbitrary node in $B_y\setminus B_x$, and replace the edge $xy$ with the path $xx'y$.  This preserves $t$-simplicity since $|B_{x'}|\le t$, and $B_{x'}$ is the only bag of $\mathcal{T}$ that contains $(B_x\cap B_y)\cup\{v,w\}$.  [TODO: Check if this paragraph is needed or if it's covered by the rest of the argument.]

    We are finally left with the case where $|B_y|=t+1$.
     % and $|B_x\cap B_y|= t-1$.
    Consider the subtree $T_y$ of $T$ containing all $T$-descendants of $y$, including $y$.  $T_y$ has a subtree $T_y'$ containing all nodes $u$ such that $B_u\supseteq B_x\cap B_y$.  The tree $T_y'$ is non-empty and therefore contains some node $u$ of degree at most one.  We replace the edge $xy$ in $T$ with $xu$.  This does not change the fact that $\mathcal{T}$ is a $t$-simple tree decomposition of $G$.  Setting $y:=z$ allows us to continue, now assuming that $y$ has at most one child $z$ with $B_z\supseteq B_x\cap B_y$.

    Since $|B_x\cap B_y|<t$ and $|B_y|=t+1$, $B_y\setminus B_x$ contains at least two elements $w$ and $\xi$.  At least one of the following is true:
    \begin{inparaenum}[(i)]
        \item $y$ has no child $z$ such that $B_z\supseteq B_x\cap B_y$;
        \item $y$ has exactly one child $z$ such that $B_z\supseteq B_x\cap B_y$ and at least one of $w$ or $\xi$, say $w$, is not contained in $B_z$.
    \end{inparaenum}
    In this case, we introduce a new node $x'$ with $B_{x'}:=(B_x\cap B_y)\cup\{v,w\}$ and replace the edge $xy$ with the path $xx'y$.  We claim that this operation preserves $t$-simplicity.  To see this, we need to check three conditions:
    \begin{compactenum}
        \item The $t$-element set $(B_x\cap B_y)\cup\{v\}$ is contained in $B_x$ and $B_{x'}$ only.
        \item The $t$-element set $(B_x\cap B_y)\cup\{w\}$ is contained in $B_{x'}$ and $B_y$ only.
        \item Any $t$-element subset of $B_{x'}$ that includes both $v$ and $w$ appears only in $B_{x'}$.
    \end{compactenum}
    Again, this operation increases the total bag size of $\mathcal{T}$, so this step is complete.
    % This operation increases the total size of all bags in $\mathcal{T}$ and therefore represents progress.
    %
    %
    %
    %     $\{v,w\}$
    %
    % % If $|
    %
    %
    %
    %
    %
    % If $T_y'$ contains a node $z$ with $|B_z|<t+1$, then we do this and the algorithm continues.  Otherwise, $T_y'$
    % %
    %
    % (Assume that $|B_x\cap B_y|=t-1$, otherwise things are easy.)
    %
    % If $|B_y|=t+1$ then, let $C:=(B_x\cap B_y)\cup \{v\}$ where $v$ is defined as above (so $v$ does not appear in the bag of the parent of $x$).  Let $\{v_1,\ldots,v_d\}:=B_y\setminus B_x$.  Since $|B_x\cap B_y|<t$, $d\ge 2$.
    %
    %
    %  There are some cases to consider:
    %
    % \begin{compactenum}
    %
    %     \item If $y$ has a child $z$ such that $v_1\cup (B_x\cap B_y)\not\subseteq B_z$, then we replace $xy$ with a path $xx'y$ where $B_{x'}:C\cup\{v_1\}$.  This preserves $t$-simplicity since the only new $t$-clique to worry about is $v_1\cup (B_x\cap B_y)$ but this only appears in $B_y$ (and now in $B_{x'}$).
    %
    %     \item Otherwise, $y$ has exactly one child $z$ such that $v_1\in B_z$.  Now, observe that $C':=(B_x\cap B_y)\cup\{v_1\}$ is contained in $B_y$ and $B_z$.  Therefore,
    %
    %
    % arguing as above, there is a vertex $v$ that appears only in $B_x$.
\end{proof}

Finally, a property of treewidth that is useful for induction \cite{kundgen.pelsmajer:nonrepetitive,bose.dujmovic.ea:asymptotically} is that, for each breadth-first-search layer $L_i$ of a graph $H$, the treewidth of $H[L_i]$ is less than the treewidth of $H$.

\begin{lem}\label{simple-bfs-layers}
    Let $H$ be a simple $t$-tree, let $n:=|H|$, let $v_1,\ldots,v_n$ be a construction order for $H$, and let $L_0,\ldots,L_m$ be the BFS ordering of $H$ with $L_0:=v_1,\ldots,v_{\min\{t,n\}}$.   Then, for each $i\in\{1,\ldots,m\}$, $\stw(H[L_i])\le t-1$.
\end{lem}

\begin{proof}
    Delete BFS layers $L_{i+1},\ldots,L_m$ and contract BFS layers $L_0,\ldots,L_{i-1}$ into a single vertex $v_0$ to obtain a minor $H'$ of $H$ in which $v_0$ is a dominant vertex.  By \cref{simple-minor-closed}, $\stw(H')\le t$, so $H'$ has a $t$-simple tree decomposition $(B_x:x\in V(T))$.

    Now, let $T'$ be obtained by removing each node $x\in V(T)$ such that $v_0\not\in B_x$ and let $\mathcal{T}':=(B_x':x\in V(T'))$, where $B_x':=B_x\setminus\{v_0\}$.  We claim that $\mathcal{T}'$ is a $(t-1)$-simple tree decomposition of $H[L_i]$.  Clearly, the width of $\mathcal{T}'$ is at most $t-1$.  $\mathcal{T}'$ is also $(t-1)$-simple since, for any distinct triple $x,y,z\in V(T')$, $|B_x\cap B_y\cap B_z|<t$ and $v_0\in B_x\cap B_y\cap B_z$, so $|B_x'\cap B_y'\cap B_z'\setminus\{v_0\}|<t-1$.

    All that remains is to verify that $\mathcal{T}'$ is, indeed, a  tree decomposition of $H[L_i]$.  Property (i) of tree decompositions (connectivity of subtrees) follows immediately from property (i) of $\mathcal{T}$.  Finally, property (ii) is satisfied since, for any edge $vw\in E(H[L_i])$, the vertices $v_0,v,w$ form a clique in $H'$ so, by \cref{TreeDecompositionClique}, there is a node $x\in V(T)$ such that $\{v_0,v,w\}\subseteq B_x$ and therefore $x\in V(T')$ so $\{v,w\}\subseteq B_x'$.
\end{proof}


Next we describe a characterization of simple treewidth in terms of subgraphs. Let $W_k$ be the graph with vertex set $\{u,v,w,x_1,\dots,x_k\}$, where each of
$\{u,x_1,\dots,x_k\}$, $\{v,x_1,\dots,x_k\}$ and $\{w,x_1,\dots,x_k\}$ are cliques.

\begin{lem}
\label{SimpleTreewidthChordal}
    A graph $G$ has simple treewidth at most $k$ if and only if $G$ is a subgraph of a chordal graph containing no $K_{k+2}$ or $W_k$ subgraph.
\end{lem}

\begin{proof}
    In every tree-decomposition of $W_k$ with width $k$, there are three distinct bags containing $\{u,x_1,\dots,x_k\}$, $\{v,x_1,\dots,x_k\}$ and $\{w,x_1,\dots,x_k\}$. Such a tree-decomposition is not $k$-simple. Thus $\stw(W_k)>k$.

    Now consider a graph $G$ with simple treewidth at most $k$ and let $\mathcal{T}:=(B_x:x\in V(T))$ be a $t$-simple tree decomposition of $G$. Let $G'$ be the graph obtained from $G$ by adding an edge between any two non-adjacent vertices in a common bag $B_x$, i.e., $V(G'):=V(G)$, $E(G'):=\bigcup_{x\in V(T)}\{vw: v,w\in B_x,\, v\neq w\}$. Observe that $\mathcal{T}$ is also a $k$-simple tree-decomposition of $G'$, so $\stw(G')\leq k$. By \cref{TreeDecompositionChordal}, $G'$ is chordal, and $W_k\not\subseteq G'$ since $\stw(W_k)>k$. By \cref{TreeDecompositionClique}, $K_{k+2}\not\subseteq G'$.

    We now prove that if $G$ is a subgraph of a chordal graph with no $K_{k+2}$ or $W_k$ subgraph, then $\stw(G)\leq k$. It suffices to prove the result when $G$ is chordal with no $K_{k+2}$ or $W_k$ subgraph. Let $\{X_i:i\in I\}$ be the maximal cliques of $G$. Let $T$ be the graph with vertex-set $I$, where $ij\in E(T)$ if and only if $X_i\cap X_j\neq\emptyset$ and $i\neq j$. It is well-known that $T$ is a tree and $(X_i:i\in V(T))$ is a tree-decomposition of $G$ (since $G$ is chordal). Since $K_{k+2}\not\subseteq G$, the width is at most $k$. If some set $X$ of $k$ vertices appears in distinct bags $X_a,X_b,X_c$, then $X_a\cup X_b\cup X_c$ induce $W_k$. Thus $(X_i:i\in V(T))$ is a $k$-simple tree-decomposition of $G$, and $\stw(G)\leq k$.
\end{proof}

[TODO: Add Wulf's characterization of $t$-trees in terms of excluded minors.]


\bibliographystyle{plainnat}
\bibliography{stw}

\end{document}
